\chapter{Keyframe Scale Recovery with GPS inclusion}
\label{sec:gliederung}

Ein Text kann mit den Befehlen \texttt{\textbackslash
chapter\{.\}}, \texttt{\textbackslash section\{.\}},
\texttt{\textbackslash subsection\{.\}} und \texttt{\textbackslash
subsubsection\{.\}} gegliedert werden.

\section{Updated Direct Monocular Slam Architecture}
\label{sec:gliederung}

Ein Text kann mit den Befehlen \texttt{\textbackslash
chapter\{.\}}, \texttt{\textbackslash section\{.\}},
\texttt{\textbackslash subsection\{.\}} und \texttt{\textbackslash
subsubsection\{.\}} gegliedert werden.

\section{Analysis}
\label{sec:gliederung}

Ein Text kann mit den Befehlen \texttt{\textbackslash
chapter\{.\}}, \texttt{\textbackslash section\{.\}},
\texttt{\textbackslash subsection\{.\}} und \texttt{\textbackslash
subsubsection\{.\}} gegliedert werden.


Ein Text kann mit den Befehlen \texttt{\textbackslash
chapter\{.\}}, \texttt{\textbackslash section\{.\}},
\texttt{\textbackslash subsection\{.\}} und \texttt{\textbackslash
subsubsection\{.\}} gegliedert werden.

\section{Results}
\label{sec:gliederung}

Ein Text kann mit den Befehlen \texttt{\textbackslash
chapter\{.\}}, \texttt{\textbackslash section\{.\}},
\texttt{\textbackslash subsection\{.\}} und \texttt{\textbackslash
subsubsection\{.\}} gegliedert werden.
