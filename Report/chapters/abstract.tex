\chapter*{Abstract}
\addcontentsline{toc}{chapter}{Abstract}
%\chapter*{Zusammenfassung}
%\addcontentsline{toc}{chapter}{Zusammenfassung}

This semester thesis describes a loosely coupled pipeline for realtime Simultaneous Localization and Mapping onboard a fixed winged aircraft for the purpose of path planning using monocular vision and raw GPS data as input. Unlike conventional sensor fusion based slam approaches for unmanned aerial vehicles, no inertial measurement data is used in this pipeline tackling the scenarios when either inertial measurement data is not available, or is rendered unusable for vison slam for any case specific reason. The approach is based on the vision-only Direct Semi-Dense Mapping framework provided by the LSD-Slam package for keyframe based odometry and depth map generation. Additionally, raw GPS data is incorporated into the SLAM pipeline through the factor-graph optimization framework provided by GTSAM package to recover metric scale and true global position and orientation of the generated pointcloud. \\
Direct featureless photometric image alignment coupled with short-baseline stereo is used to generate a pose graph of keyframes with associated depth and variance, connected via Sim3 transition edges incorporating both scale and pose transformation between the keyframes. Robust error norms are used on the Sim3 transition edges to supress the effect of outliers in the optimization. GPS measurements are used to tag each keyframe with their global position estimate as obtained from the raw GPS data with it's associated uncertainty. The thus created graph with similarity factors from visual odometry and point factors from GPS is optimized to obtain the global pose and scale estimate of each keyframe and it's associated depth map, constituting the final pointcloud.\\
The pipeline is run on a standard PC as a node in the ROS framework, and was tested offline using data that was previously recorded in AtlanticSolar’s flight over Marche-en-Famenne, Belgium as a part of ICRAUS field trials.