\chapter{Introduction}
\label{sec:introduction}
%\chapter{Einleitung}
%\label{sec:einleitung}
Unmanned Aerial Vehicles, both fixed and rotory winged, have found plathora of applications in a wide variety of fields in recent years. Precision agriculture, environmental and  wildlife monitoring, search and rescue missions are but a few examples where UAVs are already making a visible impact. A more denser and seemless integration of unmanned aerial vehicles, especially autonomous unmanned aerial vehicles flying beyond line of sight, in day to day life poses a variety of challanges, both social and scientific. On the technical side, it requires UAVs to have a better, more richer understanding of the environmemnt they operate in. Apart from developing a higher level of cognitive understanding of the UAV's surroundings, still an open research problem, it is vitally important for the UAV to be able to create a simplified map of it's surroundings and use it for navigate across obstacles and avoid collisions. Thus, Simultaneous Localization and Mapping for autonomous unmanned aerial vehicles flying in unknown environments has garnered a lot of research interest. The ability to create a map of a UAV's surrounding environment in realtime using onboard sensors and estimate it's location in relation to the map enables the possibility of online path planning for collision avoidance in unknown environments.\\
At the top level, it is this problem of building a map of a fixed winged UAV's immediate surroundings and localizing the UAV within the map, onboard and in realtime, for the purpose of navigation and collision avoidance is what is delt with in this semester thesis. In the following sections, the specific SLAM problem being tackled in the thesis is motivated and the system's specifications are described, followed by problem formulation and an outline of the solution approach.



\section{Motivation}
\label{sec:intro_motivation}
Fixed wing UAVs are more efficient in forward flight than their rotary wing conterparts, having to spend energy only to counter the aerodynamic drag(in addition to powering electronics) and using aerodynamic lift to balance weight, in contrast to rotary wing UAVs that spend energy to counter both weight and drag. This makes fixed wing UAVs ideally suited for long range or high endurance missions such as search and rescue operations in outdoor environments. The requirement to operate autonomously or semiautonomously in such unknown or unmapped terrains for these missions makes onboard mapping a necessity for safe and reliable operation. Of the variety of information rich onboard extereoceptive sensors that have been studied for their application in SLAM, including range sensors like Laser scanners and vision sensors like RGB-D cameras, stereo and monocular cameras, the payload's weight and energy constraints on small UAVs poses significant restrictions on the kind of sensors that can be used on such small airborn vehicles.\\ 
Compact, light weight and low cost sensors like monocular cameras providing information rich snapshots of the surroundings deliver many advantages from the point of view of UAV's payload and energy requirements. At the same time, they also pose significant challanges in terms of their application for SLAM. Unlike range sensors like laser scanners, monocular vision sensors do not provide direct measurements of the 3D coordinates of the terrain, but require it to be extracted from the sequence of images they provide. Similarly, while laser range sensors and both RGB-D cameras and stereo vision sensors can make measurements with an associated metric scale, scale by itself is not observable in measurements made by a monocular vision sensor. That is, camera translation and scene depth can both be changed such that the measurements made by the monocular camera remains constant. This requires the fusion of sensor measurements made by some other mertic sensor onboard the aircraft with the measurements from the monocular camera to both counter scale drift and recover the global scale and orientation of the map.\\
Since the purpose of mapping is to enable a path planning procedure for collision avoidance, building and maintaining a map only of the UAV's local surroundings is deemed sufficient, rather than maintaining a map over the entire history of the UAV's mission, in light of the expected long range and exploratory nature of mission and limited computational capabilities of the onboard computer. This motivates the semster thesis to deal with real-time mapping over a sliding window on a fixed wing UAV for the purpose of navigation and collision avoidance using a monocular vision sensor. Details of the aircraft, it's various sensors and the flight mission are described in the following section.

\section{System Description}
\label{sec:intro-sys_description}

Ein Text kann mit den Befehlen \texttt{\textbackslash
chapter\{.\}}, \texttt{\textbackslash section\{.\}},
\texttt{\textbackslash subsection\{.\}} und \texttt{\textbackslash
subsubsection\{.\}} gegliedert werden.

\section{Problem Statement}
\label{sec:intro_prob_statement}

Ein Text kann mit den Befehlen \texttt{\textbackslash
chapter\{.\}}, \texttt{\textbackslash section\{.\}},
\texttt{\textbackslash subsection\{.\}} und \texttt{\textbackslash
subsubsection\{.\}} gegliedert werden.

\section{Solution Approach}
\label{sec:intro_sol_approach}
