\chapter{Introduction}
\label{sec:introduction}
%\chapter{Einleitung}
%\label{sec:einleitung}
Unmanned Aerial Vehicles, both fixed and rotory winged, have found plathora of applications in a wide variety of fields in recent years. Precision agriculture, environmental and  wildlife monitoring, search and rescue missions are but a few examples where UAVs are already making a visible impact. A more denser and seemless integration of unmanned aerial vehicles, especially autonomous unmanned aerial vehicles flying beyond line of sight, in day to day life poses a variety of challanges, both social and scientific. On the technical side, it requires UAVs to have a better, more richer understanding of the environmemnt they operate in. Apart from developing a higher level of cognitive understanding of UAV's surroundings, still an open research problem, it is important for the UAVs to at least be able to create a simplified map of it's surroundings and use it for navigate across obstacles and avoid collisions. Thus, Simultaneous Localization and Mapping for autonomous unmanned aerial vehicles flying in unknown environments has garnered a lot of research interest. The ability to create a map of a UAV's surrounding environment in realtime using onboard sensors and estimate it's location in relation to the map enables the possibility of online path planning for collision avoidance in unknown environments.\\
At the top level, it is this problem of building a map of a fixed winged UAV's immediate surroundings and localizing the UAV within the map, onboard and in realtime, for the purpose of navigation and collision avoidance is what is delt with in this semester thesis. In the following sections, the specific SLAM problem being tackled in the thesis is motivated and the system's specifications are described, followed by problem formulation and an outline of the solution approach.



\section{Motivation}
\label{sec:intro_motivation}

A varienty of information rich onboard extereoceptive sensors have been studied for their application in SLAM, including range sensors like Laser scanners and vision sensors like RGB-D cameras, stereo and monocular cameras. 

Ein Text kann mit den Befehlen \texttt{\textbackslash
chapter\{.\}}, \texttt{\textbackslash section\{.\}},
\texttt{\textbackslash subsection\{.\}} und \texttt{\textbackslash
subsubsection\{.\}} gegliedert werden.

\section{System Description}
\label{sec:intro-sys_description}

Ein Text kann mit den Befehlen \texttt{\textbackslash
chapter\{.\}}, \texttt{\textbackslash section\{.\}},
\texttt{\textbackslash subsection\{.\}} und \texttt{\textbackslash
subsubsection\{.\}} gegliedert werden.

\section{Problem Statement}
\label{sec:intro_prob_statement}

Ein Text kann mit den Befehlen \texttt{\textbackslash
chapter\{.\}}, \texttt{\textbackslash section\{.\}},
\texttt{\textbackslash subsection\{.\}} und \texttt{\textbackslash
subsubsection\{.\}} gegliedert werden.

\section{Solution Approach}
\label{sec:intro_sol_approach}
