\chapter{Direct Monocular Slam with Pose Graph Optimization}
\label{sec:lsdslam}
%\chapter{Einleitung}
%\label{sec:einleitung}

Hier kommt die Einleitungc1


Nachfolgend wird die Codierung einiger oft verwendeten Elemente
kurz beschrieben. Das Einbinden von Bildern ist in \LaTeX\ nicht
ganz unproblematisch und hängt auch stark vom verwendeten Compiler
ab. Typisches Format für Bilder in \LaTeX\ ist
EPS\footnote{Encapsulated Postscript} oder PDF\footnote{Portable Document Format}.


\section{Introdduction to LSD SLAM}
\label{sec:gliederung}

Ein Text kann mit den Befehlen \texttt{\textbackslash
chapter\{.\}}, \texttt{\textbackslash section\{.\}},
\texttt{\textbackslash subsection\{.\}} und \texttt{\textbackslash
subsubsection\{.\}} gegliedert werden.

\section{Methodology}
\label{sec:refverw}

Literaturreferenzen werden mit dem Befehl \texttt{\textbackslash
citep\{.\}} und \texttt{\textbackslash
citet\{.\}} erzeugt. Beispiele: ein Buch \citep{Raibert1986LeggedRobotsThatBalance}, ein Buch und ein Journal Paper \citep{Raibert1986LeggedRobotsThatBalance,Vukobratovic2004ZeroMomentPoint}, ein Konferenz Paper mit Erwähnung des Autors: \citet{Pratt1995SEA}.

\subsection{Tracking}
\label{sec:refverw}

Literaturreferenzen werden mit dem Befehl \texttt{\textbackslash
citep\{.\}} und \texttt{\textbackslash
citet\{.\}} erzeugt. Beispiele: ein Buch \citep{Raibert1986LeggedRobotsThatBalance}, ein Buch und ein Journal Paper \citep{Raibert1986LeggedRobotsThatBalance,Vukobratovic2004ZeroMomentPoint}, ein Konferenz Paper mit Erwähnung des Autors: \citet{Pratt1995SEA}.


\subsection{Keyframe Depth Estimation}
\label{sec:refverw}

Literaturreferenzen werden mit dem Befehl \texttt{\textbackslash
citep\{.\}} und \texttt{\textbackslash
citet\{.\}} erzeugt. Beispiele: ein Buch \citep{Raibert1986LeggedRobotsThatBalance}, ein Buch und ein Journal Paper \citep{Raibert1986LeggedRobotsThatBalance,Vukobratovic2004ZeroMomentPoint}, ein Konferenz Paper mit Erwähnung des Autors: \citet{Pratt1995SEA}.

\subsection{Map Optimization}
\label{sec:refverw}

Literaturreferenzen werden mit dem Befehl \texttt{\textbackslash
citep\{.\}} und \texttt{\textbackslash
citet\{.\}} erzeugt. Beispiele: ein Buch \citep{Raibert1986LeggedRobotsThatBalance}, ein Buch und ein Journal Paper \citep{Raibert1986LeggedRobotsThatBalance,Vukobratovic2004ZeroMomentPoint}, ein Konferenz Paper mit Erwähnung des Autors: \citet{Pratt1995SEA}.

Zur Erzeugung von Fussnoten wird der Befehl \texttt{\textbackslash
footnote\{.\}} verwendet. Auch hier ein Beispiel\footnote{Bla
bla.}.

Querverweise im Text werden mit \texttt{\textbackslash label\{.\}}
verankert und mit \texttt{\textbackslash cref\{.\}} erzeugt.
Beispiel einer Referenz auf das zweite Kapitel:
\cref{sec:latexumg}.


\section{Results and Observation}
\label{sec:aufz}

Folgendes Beispiel einer Aufzählung ohne Numerierung,
\begin{itemize}
  \item Punkt 1
  \item Punkt 2
\end{itemize}
wurde erzeugt mit:
\begin{verbatim}
\begin{itemize}
  \item Punkt 1
  \item Punkt 2
\end{itemize}
\end{verbatim}

Folgendes Beispiel einer Aufzählung mit Numerierung,
\begin{enumerate}
  \item Punkt 1
  \item Punkt 2
\end{enumerate}
wurde erzeugt mit:
\begin{verbatim}
\begin{enumerate}
  \item Punkt 1
  \item Punkt 2
\end{enumerate}
\end{verbatim}

Folgendes Beispiel einer Auflistung,
\begin{description}
  \item[P1] Punkt 1
  \item[P2] Punkt 2
\end{description}
wurde erzeugt mit:
\begin{verbatim}
\begin{description}
  \item[P1] Punkt 1
  \item[P2] Punkt 2
\end{description}
\end{verbatim}

